%%%%%%%%%%%%%%%%%%%%%%%%%%%%%%%%%%%%%%%%%%%%%%%%%%%%%%%%%%%%%%%%%%%%%%%%%%%%%%%%
% ABSTRACT
\begin{resumo}

\hypertarget{estilo:resumo}{}

A hybrid metaheuristic combing the Multi-Particle Collision Algorithm (MPCA) with the Hooke-Jeeves (HJ) method is applied to identify the structural damage. A new version of the MPCA is formulated with the rotation-based learning mechanism to the exploration search. The inverse problem of damage identification is formulated as an optimization problem assuming the displacement time history as experimental data. The objective function is the square difference between the measured displacement and the displacement calculated by the forward model. The approach was tested on a cantilevered beam structure. Time-invariant damages were assumed to generate the synthetic displacement data. Noiseless and noisy data were considered. Finite element method was used for solving the direct problem.  The comparison between standard MPCA-HJ and the new version of the hybrid method is reported. The use of these hybrid algorithms allows to obtain good estimations using a full set of data, or using a reduced dataset with a low level of noise in data.


The hybrid metaheuristic Rotation-Based Sampling Multi-Particle Collision Algorithm with Hooke-Jeeves (RBSMPCA-HJ) is applied for damage identification in the Kabe's problem. Multi-Particle Collision Algorithm is a metaheuristic algorithm that performs a search on the solution space. With the addition of the Rotation-Based Sampling mechanism to the exploration search, a major area of the solution space has a chance to be visited. The Hooke-Jeeves is a direct search method that exploits the best solution found by the RBSMPCA, allowing to achieve better solutions. Experimental data were generated in silico, using time-invariant damages. Experiments with noiseless and noisy data were carried out. Good estimations of damage location and severity are achieved.

\palavraschave{%
	\palavrachave{vibration-based damage identification}%
	\palavrachave{nastran}%
	\palavrachave{inverse problem}%
	\palavrachave{multi-particle collision algorithm}%
	\palavrachave{qG method}%
	\palavrachave{hooke-jeeves}%
}
 
\end{resumo}

\begin{abstract}

\selectlanguage{portuguese}

\hypertarget{estilo:abstract}{}

Resumo em português

\keywords{%
	\palavrachave{Identificação de danos baseada em vibração}%
	\palavrachave{nastran}%
	\palavrachave{problema inverso}%
	\palavrachave{Algoritmo de colissão de multiplas partículas}%
	\palavrachave{método qG}%
	\palavrachave{hooke-jeeves}%
}

\selectlanguage{english}

\end{abstract}